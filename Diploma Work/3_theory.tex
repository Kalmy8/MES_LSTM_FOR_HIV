\chapter{Теоретическая часть}
\label{ch:theory}

\section{Краткая справка о ВИЧ}
\linespread{1.5}
    Вирус имуннодефицита человека (сокращенно ВИЧ), представленный в виде двух типов (ВИЧ-1 и ВИЧ-2), относится к семейству ретровирусов, т.е. вирусов, встраивающих копию своей РНК (генома) в геном клетки живого организма. ВИЧ поражает клетки имунной системы, связываясь с CD4 рецепторами на их поверхности.

    Если связывание происходит успешно, вирусная частица затем проникает внутрь клетки, и запускает процесс обратной транскрипции, то есть переводит собственную генетическую информацию, записанную в виде РНК, в привычную для нашего организма ДНК.

    После того, как процесс обратной транскрипции завершен, синтезированная вирусом ДНК встраивается в ДНК клетки организма при помощи белка, называемого интегразой. 

    Инфицированная клетка начинает производить новые экземпляры вирусных белков, из которых
    впоследствии формируются новые вирусные частицы, распространяющиеся по организму, сама клетка при этом спустя время погибает.

    Таким образом, вирус ВИЧ, попав в организм, начинает распространяться среди клеток имунной системы, уничтожая их со временем, что приводит к критическим нарушениям работы имунной системы организма, вплоть до состояния, когда любая внешняя инфекция (бактериальная, вирусная, грибковая или паразитическая), становится смертельно опасной для человека.

    Антиретровирусная терапия, в свою очередь, направленна на замедление или блокирование каждого из этапов, описанных выше. Человек, принимающий такую терапию, способен полностью нивелировать влияние ВИЧ-инфекции на клетки имунной системы. Тем не менее, вирусные частицы продолжают свое нахождение в организме, поэтому терапии необходимо придерживаться в течение всей жизни.


\section{Постановка задачи машинного обучения}
\linespread{1.5}
    Цель данной исследовательской работы подразумевает получение  прогноза заболеваемости ВИЧ-инфекции на ближайший год для каждого из субъектов Российской Федерации.

    В области машинного обучения подобную задачу (задачу о предсказании некоторого числа) принято называть задачей регрессии. В общем случае решение подобной задачи сводится к извлечению зависимости исследуемой переменной от всех прочих переменных в процессе обработки некоторого массива данных при помощи алгоритма машинного обучения. Сам процесс извлечения зависимости называется обучением алгоритма.
    Когда обучение завершено, используя актуальный набор данных и усвоенную алгоритмом зависимость, можно построить дальнейший прогноз.

    Таким образом, для решения поставленной задачи регрессии, нам необходимо выбрать некоторый алгоритм машинного обучения, а так же собрать и подготовить данные для обучения и финального предсказания.

    В данной работе в качестве алгоритма было принято использовать несколько разновидностей нейронных сетей, популярных для работы с временными последовательностями. Хорошо известно, что нейронные сети способны эффективно вычленять из данных закономерности любой сложности и в сочетании с классическими статистическими методами обеспечивают лучшее качество прогнозирования временных рядов \cite{M4, Multivariate_ES_LSTM_best}.

    Также было принято решение обогатить информацию о заболеваемости ВИЧ-инфекцией за прошедшие года дополнительной социально-демографической информацией, которая, по результатам исследований \cite{UrFu_socio_factors, Zanakis_Ortiz_socio_factors}, связана с распространением ВИЧ. Прогнозирование временного ряда с учетом дополнительных параметров принято называть прогнозированием многомерного ряда. Такой подход, судя по некоторым исследованиям, способен улучшить качество предсказания \cite{Multivariate_over_Univariate, VARMA_better_then_univariate}. 
    
    Вместе с тем, необходимо отметить, что в некоторых исследованиях тот же подход показал худший или сравнимый результат с более простыми моделями, которые не учитывают внешние факторы \cite{Univariate_better, Univariate_better_2}. По всей видимости, возможная выгода от использования дополнительных факторов зависит от конкретной решаемой задачи. 
    
    Существует по крайней мере одно исследование \cite{RFOREST_HIV_predict}, использующее социально-демографические факторы как предикторы для классификации пациентов по двум группам исходя из их ВИЧ-статуса (ВИЧ-положительные и ВИЧ-отрицательная группа). В данном исследовании авторы использовали другой алгоритм машинного обучения, так называемый <<Случайный лес>>, который с точностью в 82,36 \%, предсказывал принадлежность пациента к первой или второй группе исходя из его социально-демографических характеристик. Результаты этой работы наталкивают на мысль о том, что использование социально-демографических предикторов может принести пользу и в рамках нашей задачи. 

    Информация о заболеваемости ВИЧ извлечена из информационных бюллетеней Роспотребнадзора \cite{Информационные_бюллетени_Роспотребнадзор}, вся прочая информация извлечена из Единой межведомственной информационно-статистической системы \cite{ЕМИСС}, которая объединяет в себе официальную государственную статистику из множества источников. 


\section{Обзорный анализ работ, посвященных применению социально-демографических данных для предсказания заболеваемости ВИЧ, выбор факторов}
\linespread{1.5}
    Использование многомерных временных рядов во многих случаях позволяет получить более точный прогноз, чем при прогнозировании без дополнительных факторов, однако для этого необходимо провести дополнительную работу по выбору и обработке дополнительных параметров.

    Для выбора дополнительных факторов были проанализированы существующие работы, устанавливающие взаимосвязь между заболеваемостью ВИЧ-инфекцией и социально-демографическими предикторами посредством их корреляции. Существует большое количество научных работ и систематических обзоров на данную тему, однако, по большей части они ограничиваются небольшим локальным регионом (когортой), в рамках которого устанавливается взаимосвязь \cite{Ethiopia_Demographic_HIV, Ethiopia_Demographic_HIV_2, Dempgraphic_impact_of_HIV}. 
    
    Наибольший интерес для нас представляет наиболее масштабное исследование учёных Стелиоса Занакиса, Сесилии Альварес и Вивиена Ли \cite{Zanakis_Ortiz_socio_factors}, объединяющее в себе результаты множества подобных работ и устанавливающее корреляцию заболеваемости ВИЧ с более чем 80 социально-демографическими факторами. Все факторы по своей природе были разбиты на несколько групп (показатели здравоохранения, экономические показатели, показатели образованности, демографические показатели и медийные показатели). Факторы, имеющие большое количество пропусков (>25 \%) или сильную скореллированность с другими факторами (Variance Inflation Factor < 6) были удалены из рассмотрения, финальное количество предикторов составило 56 штук. Затем, с применением линейной регрессии, авторы выделили 6 наиболее статистически значимых факторов, объясняющих показатель заболеваемости ВИЧ-инфекцией. Выявленные зависимости приведены в таблице \ref{tab:Zanakis-table}.

% Please add the following required packages to your document preamble:
% \usepackage{graphicx}
\begin{table}[h]
\caption{Значимые предикторы в глобальной модели ВИЧ/СПИД \\ Источник: {[}S.H. Zanakis et al. / European Journal of Operational Research 176 (2007) 1811–1838{]}}
\label{tab:Zanakis-table}
\resizebox{\textwidth}{!}{%
\begin{tabular}{ll}
\hline
Объясняемая переменная &
  \begin{tabular}[c]{@{}l@{}}Заболеваемость ВИЧ \\ на 100 тыс. чел.\end{tabular} \\ \hline
\textit{Здоровье}                                   &     \\
Индекс производительности системы здравоохранения   &     \\
Расходы на здравоохранение на душу населения по ППС & -   \\
Собственные расходы граждан на здравоохранение ППС  &     \\
Количество больничных коек на 1000 человек          &     \\
Количество врачей на 100 000 человек                & -Ln \\
Количество медсестер на 100 000 человек             &     \\
\textit{}                                           &     \\
\textit{Благосостояние}                             &     \\
Коммерческое потребление энергии ВВП                & +   \\
Чистый импорт коммерческой энергии                  &     \\
Валовой национальный продукт                        &     \\
                                                    &     \\
\textit{Демографические данные}                     &     \\
Процент экономически зависимых возрастов 15–59 лет  & +   \\
Естественный прирост населения                      &     \\
Плотность населения                                 &     \\
Уровень рождаемости                                 & -   \\
Доля занятых в сельском хозяйстве (\%)              & -Ln \\
                                                    &     \\
\textit{Доступность СМИ}                            &     \\
Радиоприемники на 1000 человек                      &     \\ \hline
\begin{tabular}[c]{@{}l@{}}Ln: указывает на объясняющую переменную, линейно преобразованную \\ с помощью логарифмической трансформации.\end{tabular} &
  
\end{tabular}%
}
\end{table}

    На основе данного исследования в разрабатываемую модель вошло несколько факторов, схожих по смыслу с приведенными выше, другая часть была отброшена:

    \begin{enumerate}
    	\item Количество врачей на 10 тыс. чел. \cite{Vrachi_per_capita}: фактор был отброшен, потому как в Российской Федерации статистика по данному показателю ведется лишь с 2021 года. 
    	  
        \item Расходы на здравоохранение на душу населения по ППС: не обнаружено официальных источников, содержащих подобный показатель с группировкой по субъектам Российской Федерации за период подходящей длины. Данные подобного показателя в ЕМИСС охватывают только период с 2008 по 2012 года. Такая короткая статистика, вероятно, не позволит улучшить качество модели.
        
        \item Коммерческое потребление энергии ВВП: в нашей модели для репрезентации состояния экономики используется более точный и локальный фактор <<Валовый региональный продукт на душу населения>> \cite{vrp_per_capita}. Использование нескольких близких по смыслу, имеющих высокую корреляцию признаков, как правило, снижает качество результирующей модели.

        \item Процент экономически зависимых возрастов 15-59 лет: заменен на близкий по смыслу показатель <<Структуры численности постоянного населения по возрастным группам>> \cite{Chislennost_po_vozrastu}. Стоит отметить, что по информации Роспотребнадзора \cite{Справка_по_ВИЧ-инфекции_в_России_Роспотребнадзор_2022} в период с 2000 по 2022 года возрастные группы, наиболее подверженные распространению ВИЧ, сильно изменились. Так, в 2000 году, около 60 \% всех новых случаев ВИЧ приходились на возрастную группу от 20 до 30 лет, в то время как на группу от 30 до 40 лет приходилось только 12 \%. В 2022 году картина изменилась противоположным образом: 40 \% новых случаев приходится на вторую (старшую) группу, и только 10 \% приходится на первую. Таким образом, возникает гипотеза о том, что демографический состав населения действительно стоит учитывать при построении прогноза.

        \item Уровень рождаемости: уровень рождаемости естественным образом отрицательно коррелирует с числом заболеваний ВИЧ, потому как абсолютно подавляющее большинство детей рождается без ВИЧ-инфекции, и, по данным Роспотребнадзора \cite{Справка_по_ВИЧ-инфекции_в_России_Роспотребнадзор_2022}, менее 1 \% случаев заболевания приходится на детей до 15 лет. Таким образом, чем выше в регионе рождаемость, тем меньше случаев ВИЧ будет приходится на душу населения, однако полезной для модели информации данная статистика не несёт.

        \item Доля занятых в сельском хозяйстве (\%): заменен на близкий по смыслу и более точный показатель <<Доля городского населения>>. Исходный фактор, по словам авторов, включен в модель для учета численности городского и сельского населений. Очевидно, что ВИЧ поражает в первую очередь жителей крупных городов, которые являются очагами для распространения инфекций. Вместе с тем, согласно информации Роспотребнадзора \cite{Доклад_Роспотребнадзор_2023}, в последнее время в России отмечается рост заболеваемости именно среди жителей сельской местности.
        
    \end{enumerate}
    
    Второе исследование, результаты которого также повлияли на выбор предикторов для нашей модели, было проведено совместными усилиями Свердловского областного центра профилактики и борьбы со СПИД и Институтом Высшей школы экономики и менеджмента Уральского федерального университета имени первого Президента России Б.Н. Ельцина в 2018 году \cite{UrFu_socio_factors}. Авторы поставили целью работы выделение социально-экономических факторов, влияющих на распространение ВИЧ-инфекции в регионах России. В процессе исследования была составлена регрессионная модель, связывающая распространение ВИЧ-инфекции во всех регионах России с данными факторами. Уровни статистической значимости использованных в модели факторов приведены в таблице \ref{tab:Urfu-table}.

% Please add the following required packages to your document preamble:
% \usepackage{graphicx}
\newpage
\begin{table}[]
\caption{Результаты оценивания модели методом наименьших квадратов (все регионы). 
Источник: {[}Подымова А.С., Тургель И.Д., Кузнецов П.Д., Чукавина К.В. / Вестник УрФУ. Серия экономика и управление. 2018. Том 17. No 2. С. 242–262{]}}
\label{tab:Urfu-table}
\resizebox{\textwidth}{!}{%
\begin{tabular}{lcc}
\hline
\multicolumn{1}{l|}{\textbf{Объясняющая переменная}} &
  \multicolumn{1}{l|}{\textbf{Коэф. влияния}} &
  \multicolumn{1}{l}{\textbf{Станд. ошибка}} \\ \hline
\multicolumn{1}{l|}{Безработица}                                                           & \multicolumn{1}{c|}{-0,015***} & 0,003 \\
\multicolumn{1}{l|}{Число новых случаев наркомании}                                        & \multicolumn{1}{c|}{0,774***}  & 0,105 \\
\multicolumn{1}{l|}{ВРП на душу населения}                                                 & \multicolumn{1}{c|}{0,028***}  & 0,007 \\
\multicolumn{1}{l|}{Ввод в действие общей площади жилых домов на душу населения}           & \multicolumn{1}{c|}{0,198***}  & 0,055 \\
\multicolumn{1}{l|}{Численность населения с денежными доходами ниже прожиточного минимума} & \multicolumn{1}{c|}{-0,003**}  & 0,001 \\
\multicolumn{1}{l|}{Численность обучающихся студентов на численность населения}            & \multicolumn{1}{c|}{-0,056***} & 0,008 \\
\multicolumn{1}{l|}{Число зарегистрированных преступлений на численность населения}        & \multicolumn{1}{c|}{0,007***}  & 0,001 \\
\multicolumn{1}{l|}{Число посещений музеев на численность населения}                       & \multicolumn{1}{c|}{-0,186***} & 0,026 \\
\multicolumn{1}{l|}{Число наблюдений}                                                      & \multicolumn{2}{c}{988}                \\
\multicolumn{1}{l|}{\textit{R-sq}}                                                         & \multicolumn{2}{c}{0,26}               \\ \hline
\multicolumn{3}{l}{\begin{tabular}[c]{@{}l@{}}Примечание: * 10 \%-й  уровень значимости, ** 5 \%-й уровень значимости, *** 1 \%-й уровень значимости\\ Источник: Составлено авторами с помощью Stata\end{tabular}}
\end{tabular}%
}
\end{table}

    Как видно из результатов таблицы, используемые авторами объясняющие переменные имеют сильную (p < 0,01) статистически значимую взаимосвязь с распространением ВИЧ-инфекции в регионах Российской Федерации, что наталкивает нас на мысль об использовании тех же объясняющих переменных и для нашей модели. Этому благоприятствует и тот факт, что в процессе исследования были задействованы те же источники данных, которые были выбраны в рамках нашего исследования, это удалось узнать, связавшись с одним из авторов работы.

     На основе данного исследования в нашу модель вошли следующие факторы: 
     \begin{itemize}
         \item безработица \cite{bezrabotiza};
         \item число новых случаев наркомании \cite{new_narcomani_per_capita};
         \item ВРП на душу населения \cite{vrp_per_capita};
         \item ввод в действие общей площади жилых домов на душу населения \cite{new_houses_percent};
         \item численность обучающихся студентов на численность населения \cite{students_do_2012, students_posle_2012};
         \item число посещений музеев на численность населения\cite{museum_visits}.
          
     \end{itemize}

     Примечание: часть приведенных выше статистик приведена в абсолютном значении, в таком случае деление на душу населения произведено вручную с использованием данных о численности населения \cite{Chislennost_naselenia}.

     Единственный не вошедший в модель фактор <<Число зарегистрированных преступлений на численность населения>> был заменён факторами <<Количество предварительно расследованных уголовных преступлений, связанных с незаконным оборотом наркотических средств>> и <<Количество выявленных административных правонарушений, связанных с незаконным оборотом наркотических средств>> \cite{Ugol_do_2016, Ugol_posle_2016, Admin_do_2015, Admin_posle_2015}. Известно, что продолжительное время основным двигателем ВИЧ-инфекции являлись потребители инъекционных наркотиков, что позволяет косвенно связать количество таких потребителей с уровнем заболеваемости ВИЧ-инфекцией. Число потребителей, в свою очередь, прямым образом сказывается на обороте (и количестве преступлений) в сфере наркобизнеса. 

     Помимо факторов, заимствованных из рассмотренных работ, в модель были также включены дополнительные данные о количестве проведенных тестов на ВИЧ и количестве тех тестов, которые показали положительный результат, для каждой из категорий населения. Эти данные также были извлечены из информационных бюллетеней Роспотребнадзора \cite{Информационные_бюллетени_Роспотребнадзор}. При обследовании на ВИЧ, каждый проведенный тест снабжается специальным кодом, позволяющим определить, по каким причинам оно было проведено. Так, например, если тест был проведен добровольно по инициативе пациента, ему будет присвоен код 101, если же на обследование был направлен потребитель инъекционных наркотиков, ему будет присвоен код 102. Зная количество проведенных и положительных тестов в каждой категории, вычислим 2 полезных знания, которые могут помочь в предсказании заболеваемости:

     \begin{enumerate}
         \item Отношение числа инфицированных к числу обследованных внутри каждой группы. Показатель позволяет понять, в какой группе риск заражения выше;
         \item Доля инфицированных в каждой группе относительно общего числа инфицированных лиц. Показатель позволяет понять, какая группа является ведущей в распространении эпидемии. 
     \end{enumerate}


     Таким образом, в финальную модель в общей сложности вошло 12 дополнительных социально-демографических фактора.

\section{Обзорный анализ работ, посвященных применению различных архитектур нейронных сетей для предсказания заболеваемости ВИЧ, выбор архитектур}
\linespread{1.5}
    Во всём мире датой изобретения первой архитектуры искусственной нейронной сети считается 1958 г., когда американский психолог Франк Розенблатт изобрел первую подобную модель, названную перцептроном. Изобретение, однако, большой популярности не имело, и интересовало по большей части узкий круг исследователей, ввиду ограниченной применимости (в силу небольших вычислительных мощностей компьютеров того времени). 
    У современников же интерес к нейронным сетям возродился вновь, когда в 2010-ом году разработанная канадским инженером Алексом Крижевски сверточная нейронная сеть <<AlexNet>> с большим отрывом завоевала 1-ое место в одном из крупнейших соревнований по распознаванию цифровых изображений ImageNet. 

    В результате данного события нейросетевые архитектуры самых разных типов (сети прямого распространения, реккурентные, сверточные сети и проч.) нашли своё применение во множестве сфер, от биоинформатики (например, в задачах о молекулярном докинге) до создания систем автопилотов для наземного транспорта. В том числе, нейросети самых разных архитектур были исследованы на возможность эффективного предсказания временных рядов \cite{CNN_forecasting, ANNs_forecasting_epidemiology, Transformets_forecasting, RNN_forecasting}. 
    
    Об успешности применения конкретной архитектуры можно судить по многочисленным соревнованиям, которые регулярно проводятся в области предсказания временных рядов и включают в себя множество (десятки тысяч) разнообразных временных рядов, уравнивая шансы на победу для каждой из соревнующихся архитектур. На основе нескольких исследований \cite{NN3, NN5, M3, M4}, анализирующих результаты каждого из крупнейших соревнований за последнее время (особенно наиболее актуального соревнования <<The М4 forecasting competition>>), об эффективности использования нейросетей в рамках данной задачи были сделаны следующие выводы:
    \begin{enumerate}
        \item Из-за ограничений по длине временных серий, сложности имплементации и избыточной сложности применяемых архитектур, нейронные сети в чистом виде плохо справляются с задачей прогнозирования временных рядов, проигрывая соревнование классическим статистическим методам, таким как экспоненциальное сглаживание или ARIMA.

        \item Методы статистики, как и методы машинного обучения, выигрывают в точности, будучи скомбинированными между собой или друг с другом. Так, 13 из 17-ти наиболее точных методов соревнования <<M4>> представляют из себя комбинации нескольких статистических методов. Победителем же стала гибридная модель, построенная на основе рекуррентной нейронной сети и экспоненциального сглаживания, названная авторами как ES-RNN.

        \item Нейронные сети, по всей видимости, лучше справляются с данными, имеющими выраженный тренд, в то время как статистические методы лучше учитывают сезонность данных.

        \item Нейронные сети, по всей видимости, лучше усваивают зависимости на длинной дистанции и хороши для предсказания сразу нескольких следующих значений временного ряда, статистические же методы больше подходят для краткосрочных прогнозов.
        
    \end{enumerate}

    Подытожив рассмотренные исследования, корректно будет сделать вывод о том, что в настоящее время нет единственного универсального метода для решения всех задач, связанных с прогнозированием. В то же время, реккурентные нейронные сети показывают свою конкурентоспособность, будучи использованны совместно со статистическими методами, особенно в задачах, связанных с долгосрочным прогнозированием.

    В то же время отметим, что все указанные выше соревнования были проведены с использованием одномерных временных рядов, то есть без использования дополнительных факторов для улучшения предсказания, и на момент написания статьи ещё не было проведено ни одного масштабного соревнования в области прогнозирования многомерных временных рядов. Существует, однако, исследование, авторы которого расширили победившую в <<M4>> соревновании модель за счёт возможности обрабатывать многомерные временные ряды, вследствие чего им удалось получить наилучший результат предсказания, превосходящий результаты использования статистических методов. Получившаяся модель носит название <<MES-LSTM>> \cite{MES_RNN}.

    Таким образом, использование рекуррентных нейронных сетей для построения предсказаний на данный момент является перспективным подходом, доказавшим свою эффективность по результатам нескольких крупных соревнований и исследований, что позволяет нам выбрать данную архитектуру в качестве основого инструмента для прогнозирования.  
    


\endinput