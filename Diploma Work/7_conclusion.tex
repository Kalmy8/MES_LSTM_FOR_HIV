\chapter*{Заключение}
\label{ch:conclusion}
\linespread{1.5}
В рамках выпускной квалификационной работы выполнены следующие задачи:

\begin{enumerate}
	\item Cобрана и изучена статистика заболеваемости ВИЧ-инфекцией в субъектах Российской Федерации, сформулирована актуальность и важность рассматриваемой проблемы.
	
	\item Произведен литературный анализ работ, посвященных социально-демографическим факторам, способствующим распространению ВИЧ-инфекции в России и в зарубежных странах, по наиболее важным факторам собрана и изучена статистика.  
	
	\item Все собранные статистики очищены от пропусков, дубликатов, аномальных значений, наименования субъектов приведены к единому стандарту, все данные агрегированы в едином носителе (таблице).
	
	\item Проведен литературный анализ работ посвященных методам прогнозирования эпидемиологических процессов, включая статистические методы и методы машинного обучения. Наиболее перспективные методы добавлены в сравнение.
	
	\item Для оценки качества прогностических моделей выбраны метрики, разработана стратегия машинного обучения.
	
	\item Выбранные модели реализованы программно, проведены эксперименты по обучению моделей и получению прогнозов, результаты представлены графически.
	
	\item В соответствии с результатами проведена работа по оптимизации параметров каждой из моделей, с использованием подобранных параметров получен финальный прогноз заболеваемости для каждого из субъектов Российской Федерации.
	
\end{enumerate}

Полученные в ходе выпускной квалификационной работы результаты позволяют судить о применимости и эффективности новейших методов машинного обучения в задачах о прогнозировании эпидемиологических процессов, извлеченные знания позволят в дальнейшем строить и оптимизировать более точные, совершенные модели.

Практическим результатом проведенной работы является прогноз заболеваемости ВИЧ-инфекции с использованием наиболее актуальных данных. С получением новых данных полученная модель может быть дообучена и использована повторно для предсказания заболеваемости ВИЧ-инфекции в последующих годах. Полученная информация далее может быть проанализирована и использована органами здравоохранения для более эффективного распределения ресурсов и проведения профилактических мероприятий в субъектах Российской Федерации. 


\endinput