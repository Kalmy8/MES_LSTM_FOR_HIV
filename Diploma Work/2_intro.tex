\chapter*{Введение}
\addcontentsline{toc}{chapter}{Введение}
\label{ch:intro}
\linespread{1.5}
    В современном мире проблема ВИЧ/СПИД остается одной из наиболее актуальных и серьезных в области здравоохранения по всему миру, в том числе в Российской Федерации. Обратимся к наиболее актуальным данным Федерального научно-методического центра по борьбе со СПИДом (Роспотребнадзора) \cite{Справка_по_ВИЧ-инфекции_в_России_Роспотребнадзор_2022} и приведем несколько ключевых положений статистики за 2022 год: 
    
\begin{enumerate}
    \item  На момент 31 декабря 2022 г. в стране проживало 1 168 076 россиян с лабораторно подтвержденным диагнозом ВИЧ-инфекции, исключая 461 879 больных, умерших за весь период наблюдения с 1987 года.
    
    \item Всего зарегистрировано 63 150 новых случаев болезни, вызванных вирусом иммунодефицита человека. Показатель заболеваемости при этом равен 43,29 на 100 тыс. населения, что на 3,8 \% больше, чем в 2021 и 2020 годах.
    
    \item На диспансерном учете состояло 835 154 больных, то есть 69,5 \% от числа россиян, живущих с диагностированной ВИЧ-инфекцией. Получали антиретровирусную терапию в 711 412 пациентов, что составляет 85,2 \% от числа состоявших на диспансерном наблюдении и 59,2 \% от общего числа живущих с диагнозом ВИЧ-инфекция.

    \item В 2022 г. было сообщено о смерти 34 410 инфицированных ВИЧ россиян, что больше чем в 2021 г. (на 0,9 \%), в 2020 г. (на 6,8 \%) и в 2019 г. (на 2,4 \%). Поскольку ВИЧ-инфекция является неизлечимым заболеванием, а число новых случаев ВИЧ-инфекции превышает число умерших, общее число россиян, живущих с ВИЧ, продолжает расти.
    
    \item По словам исследователей, <<Эпидемия ВИЧ-инфекции продолжает развиваться, и кроме увеличения охвата ВИЧ-позитивного 
    населения лечением, необходимо развитие комплекса программ по предотвращению заражения ВИЧ>>.    
    
\end{enumerate}


    Помимо высоких показателей смертности и заболеваемости, ВИЧ-инфекция также характеризуется социальной значимостью, связанной с высокой степенью стигматизации и дискриминации носителей инфекции. Некоторые люди до сих пор считают ВИЧ-инфекцию <<заслуженной>> болезнью, которая угрожает только наркоманам и людям, ведущими беспорядочную половую жизнь. Корнями эти предубеждения уходят во времена возникновения первых случаев ВИЧ в СССР, когда подобные мнения были распространены в СМИ и поддерживались местными органами власти. Так, В 1986 г. министр здравоохранения РСФСР Николай Трубилин в программе «Время» заявил:

    «В Америке СПИД бушует с 1981 года, это западная болезнь. У нас нет базы для распространения этой инфекции, так как в России нет наркомании и проституции»\cite{ВИЧ_СССР_цитата_Министра}.

    Чтобы проиллюстрировать отношение к ВИЧ в современной России, обратимся к результатам социальных опросов, проведенных в 2015 году среди 256 студентов-первокурсников в рамках научно-исследовательской работы <<ВИЧ-инфекция как социальная проблема>> \cite{Вич_социальная_проблема2015}:
    
\begin{enumerate}

    \item  38,3 \% опрошенных считают необходимым изолировать себя/своих детей от общения с ВИЧ-инфицированными людьми (избегать общения/перевести ребенка в другой класс и т.д.), 14,4 \% затрудняются ответить на данный вопрос.
    \item  9 \% опрошенных считают, что диагноз ВИЧ необходимо скрывать от друзей и знакомых, 55 \% затрудняются ответить на данный вопрос.
    \item  49 \% опрошенных считают ВИЧ-инфицированных людей асоциальными в той или иной степени.
    \item Всего 11,3 \% опрошенных узнали о ВИЧ-инфекции из уст родителей, остальные получили информацию из СМИ (66,4 \%), интернета (5,9 \%), со слов сверстников (4,3 \%) и из прочих источников.
\end{enumerate}

    Точной статистики о числе случаев дискриминации ВИЧ-инфицированных в России нет, однако, согласно информации из разных источников \cite{Современная_российская_проблематика,spid_center_2862_discrimination, spid_center_1458_discrimination}, такие люди часто сталкиваются с незаконным (ч.1 ст. 137 УК РФ) разглашением их диагноза третьими лицами, в том числе друзьями и медицинскими сотрудниками, трудностями в работе и учебе, связанными с социальной депривацией, отказом от предоставления медицинских услуг со стороны врачей, недостаточно осведомленных о путях передачи ВИЧ-инфекции и преследованиям по ст. 122 УК РФ, предполагающей уголовную ответственность за <<Заведомое поставление другого лица в опасность заражения ВИЧ-инфекцией>>.

    Социальная стигматизация не только негативно сказывается на психике ВИЧ-инфицированных людей, способствуя их выпадению из социума, но и является серьезным механизмом, сдерживающим эффективность мер по борьбе с ВИЧ-инфекцией.
    
    Так, со слов директора группы региональной поддержки для Восточной Европы и Центральной Азии (ЮНЭЙДС), главы команды ООН в России Виней Салданы, от 20 \% до 33 \% ВИЧ-инфицированных людей в странах Восточной Европы боятся обращаться в медицинские учреждения для получения терапии \cite{spid_center_1458_discrimination}. Эта оценка также согласуется с приведенными выше данными о том, что только 69,5 \% всех людей с диагнозом ВИЧ в России встают на диспансерный учет.

    Таким образом, даже если в государстве возникнет возможность помочь каждому человеку, обратившемуся за терапией, около 30 \% всех инфицированных не воспользуются этой возможностью и будут распространять заболевание дальше. 
    
    Только масштабные мероприятия по ликвидацию безграмотности населения в отношении инфекции помогут справиться с социальным угрозами ВИЧ, возникает важная задача информирования подрастающего поколения о ВИЧ и прочих болезнях, передающихся половым путем.

    Обратим внимание также и на экономический аспект значимости ВИЧ-инфекции. Поскольку заболевание до сих пор является неизлечимым (без учета единичных случаев исцеления, связанных с недоступной массово операцией по пересадке стволовых клеток костного мозга), инфицированные лица вынуждены в течение всей оставшейся жизни проходить дорогостоящую терапию антиретровирусными препаратами. 
    
    Помимо стоимости самих препаратов, стоит учитывать затраты на проведение диагностики и оказание медицинской помощи, а также косвенные экономические затраты, связанные с преждевременной смертностью и инвалидизацией трудоспособного населения.
    Согласно информации из Государственного доклада Роспотребнадзора <<О состоянии эпидемиологического благополучия населения Российской Федерации в 2022 году>> \cite{Доклад_Роспотребнадзор_2023}, общий экономический ущерб, связанный с ВИЧ-инфекцией в 2022 г. в Российской Федерации можно оценить в 262,5 млрд рублей.  

    В то же время, основываясь на информации из множества СМИ 
    \cite{forbes_deficit, gazeta_ru_Deficit, ria_ru_deficit, spid_center_3651_deficit}, в Российской Федерации нет возможности обеспечить необходимыми препаратами каждого нуждающегося в антиретровирусной терапии. Главной причиной дефицита препаратов выступает недостаток федеральных и региональных финансовых средств, выделенных для закупок, а также ошибки при планировании объема закупок. 
    
    Так, согласно доклада, подготовленного Коалицией по готовности к лечению \cite{Мониторинг_закупок_препаратов_ITPC}, в 2022 году, вследствие дефицита бюджета, 21 \% от общей суммы, затраченной на закупку антиретровирусных препаратов, был восполнен из федеральных средств, выделенных на 2023 год. Из-за допущенной ошибки, согласно СМИ \cite{ngs_ru_deficit_2023,fontanka_deficit}, в 2023 году дефицит препаратов мог составлять 50-60 \%.

    В условиях ограниченности средств, выделяемых на закупку антиретровирусных препаратов, особенно важной становится задача рационального перераспределения их в пользу тех регионов, где заболеваемость выше. Прогнозирование заболеваемости ВИЧ в отдельности для каждого субъекта Российской Федерации поможет спланировать количество приобретаемых препаратов для каждого региона с учетом эпидемиологической обстановки.
    
    Главными инструментами эпидемиологии в задачах о предсказании заболеваемости на данный момент являются динамические математические модели (SIR), разработанные в середине прошлого столетия и использующиеся повсеместно, в том числе и для ВИЧ-инфекции \cite{Лопатин_про_SIR,Sokolova_SIR_HIV,Brauer_про_SIR,Huppert_Katriel_SIR}. Однако, согласно новейшим исследованиям \cite{LSTM_for_HIV, NN_REVIEW_for_HIV, NN_FORECASTING_for_HIV}, эффективное прогнозирование заболеваемости ВИЧ требует использования современных методов анализа данных, в том числе технологий, связанных с применением нейронных сетей. 

    Использование нейросетей для прогнозирования заболеваемости ВИЧ представляет собой перспективный подход, который может значительно улучшить результаты прогнозирования эпидемиологической ситуации. Преимуществом такого подхода является дополнительный учет социально-демографических факторов, который не может быть реализован в рамках SIR-моделей. Существуют исследования, подтверждающие статистическую связь таких факторов с риском заболевания ВИЧ-инфекцией \cite{UrFu_socio_factors, Zanakis_Ortiz_socio_factors}. 
    
    Целью этой выпускной квалификационной работы является разработка нейросетевой модели для прогнозирования заболеваемости ВИЧ в субъектах Российской Федерации на основе данных о заболеваемости за прошедшие года и социально-демографических данных. Для достижения цели были поставлены следующие задачи:

    \begin{enumerate}
        \item Изучение наличного временного ряда с данными о заболеваемости за предыдущие года, выявление характерных паттернов, аномальных значений.

        \item Анализ существующих работ по прогнозированию заболеваемости ВИЧ в разных регионах мира, выявление социально-демографических факторов-предикторов роста/снижения заболеваемости.

        \item Сбор и очистка данных о социально-демографических факторах, а также данных о заболеваемости за предыдущие года, приведение всех данных к единому носителю (таблице), который будет использован для обучения нейросетей и получения прогноза.

        \item Анализ существующих работ с применением нейронных сетей в области эпидемиологии с целью поиска наиболее удачных архитектур, мотивации к их использованию.

        \item Разработка и инициализация первоначального вида нейросетей наиболее релевантных архитектур, выбор метрик качества для оценки предсказания, отбор наиболее удачных моделей.

        \item Конечная оптимизация моделей, отобранных на предыдущем шаге, получение финального прогноза заболеваемости ВИЧ-инфекции для каждого субъекта Российской Федерации.


    \end{enumerate}
    
    Объектом исследования является эпидемия ВИЧ/СПИД, а предметом - методы прогнозирования заболеваемости с использованием нейронных сетей.
   
    Практическая значимость данной работы заключается использования прогноза полученной модели для разработки более точных и эффективных стратегий борьбы с распространением ВИЧ/СПИД в Российской Федерации, предотвращения демографических и экономических убытков, связанных с эпидемией ВИЧ, и улучшения качества медицинской помощи пациентам, страдающим этим заболеванием, что представляет актуальную и важную задачу для общественного здоровья.

    Кроме того, оценка качества полученных прогнозов позволить судить о применимости используемого метода в задачах эпидемиологии, связанных с предсказанием заболеваемости, что вносит вклад в дальнейшее исследование проблемы. 

\endinput